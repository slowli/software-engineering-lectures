\documentclass{a4beamer}
%% Lectures - common definitions

\usextensions{tikz}
\usetikzlibrary{shapes.multipart,shapes.callouts,shapes.geometric}
\input{fix-callouts.inc} % Fixes absolute positioning of rectangle callouts

\newif\ifbigpages \bigpagesfalse
\ifdim\paperwidth >20cm
	\bigpagestrue
\fi

\tikzset{%
	note/.style={rectangle callout,draw=none,callout pointer width=1em,%
		align=flush left,font=\footnotesize,inner sep=0.5em,%
		fill=blue!15,fill opacity=0.95,text opacity=1.0,callout absolute pointer=#1},
	node distance=2em and 2.75em
}
\ifbigpages
	% Scale all arrow tips by the factor of 2.5
	\let\old@pgf@arrow@call=\pgf@arrow@call
	\def\pgf@arrow@call#1{%
		\@tempdima=\pgflinewidth%
		\pgfsetlinewidth{2.5\pgflinewidth}%
		\old@pgf@arrow@call{#1}%
		\pgfsetlinewidth{\@tempdima}%
	}
	\def\pgfarrowsleftextend#1{\pgfmathsetlength{\pgf@xa}{1.5*#1}}
	\def\pgfarrowsrightextend#1{\pgfmathsetlength{\pgf@xb}{1.5*#1}}
\fi

%% Explanation for a figure
\def\figureexpl#1{%
	{\linespread{1.0}\footnotesize\color{gray}{#1}\par}
}
%% English term
\def\engterm#1{(англ. \textit{#1})}
%% Term with explanation below (to be used in diagrams)
\def\termwithexpl#1#2{#1\strut{}\\\small\color{gray}(\textit{#2})\strut{}}
%% External link
\def\extlink#1#2{\href{#1}{\color[rgb]{0.7,0.7,1.0}\dashbar{#2}}}
%% Internal link
\def\inlink#1#2{\hyperlink{#1}{\color[rgb]{0.7,0.7,1.0}\dashbar{#2}}}
%% Explanation for a list item
\def\itemexpl#1{\begingroup\linespread{1.0}\small\vspace{0.75ex}#1\par\endgroup}




\lecturetitle{Программная инженерия. Лекция №2 — дисциплины ПИ.}
\title[Дисциплины ПИ]{Дисциплины программной инженерии}
\author{Алексей Островский}
\institute{\small{Физико-технический учебно-научный центр НАН Украины}\vspace{2ex}}
\date{26 сентября 2014 г.}

\begin{document}
	\frame{\titlepage}
	
	\section[]{Содержание курса}
	
	\frame{
		\frametitle{План курса программная инженерия — осенний семестр}
		
		\begin{enumerate}
			\item
			Дисциплины программной инженерии.
			\item
			Области ядра знаний SWEBOK.
			\item
			Стандарт и модели жизненного цикла ПО.
			\item
			Требования к программным системам.
			\item
			Методы объектного анализа и моделирования.
			\item
			Прикладные и теоретические методы программирования.
			\item
			Методы доказательства, верификации и тестирования ПО.
		\end{enumerate}
		
		\textbf{Форма отчетности:} зачет.
	}

	\frame{
		\frametitle{План курса программная инженерия — весенний семестр}

		\begin{enumerate}
			\item
			Интерфейсы, взаимодействие, эволюция программ и данных.
			\item
			Модели качества и надежности программных систем.
			\item
			Методы управления программным проектом. 
			\item
			Проблематика сборочного программирования программных систем.
			\item
			Инженерия производства программных продуктов. Компоненты повторного использования.
			\item
			Фабрики программ.
			\item
			Облачные и распределенные вычисления.
		\end{enumerate}

		\textbf{Форма отчетности:} экзамен.
	}
	
	\frame{	
		\frametitle{Функции программной инженерии}

		\begin{center}
		Программная инженерия = разработка программ + индустриальный подход.
		\end{center}
		
		\textbf{Функции программной инженерии:}
		\begin{itemize}
			\item
			изучение методов и средств построения компьютерных программ;
			\item
			отображение закономерностей развития и обобщение накопленного опыта прикладного программирования;
			\item
			определение автоматизированных операций по производству объектов 
			(модулей, компонентов, программных аспектов и~т.\,п.);
			\item
			определение правил и порядка инженерной деятельности для построения из простых объектов новых, более сложных 
			(программных систем, семейств систем, проектов и~т.\,п.);
			\item
			формализация методов измерения и оценки готовых программных продуктов.
		\end{itemize}
	}
	
	\frame{
		\frametitle{Дисциплины программной инженерии}
	
		\begin{tikz*}[%
	every node/.style={rectangle,draw,align=center,minimum width=7.5em}
]
	\node(se) [rectangle,draw] {Программная \\ инженерия};
	\node(sci) at (90:7.5em) [rectangle,draw,anchor=south] {Научная \\ дисциплина};
	\node(eng) at (18:7.5em) [rectangle,draw,anchor=south west] {Инженерная \\ дисциплина};
	\node(prod) at (162:7.5em) [rectangle,draw,anchor=south east] {Производственная \\ дисциплина};
	\node(ec) at (-54:7.5em) [rectangle,draw,anchor=north west] {Экономическая \\ дисциплина};
	\node(man) at (-126:7.5em) [rectangle,draw,anchor=north east] {Дисциплина \\ управления};
	
	\draw[->] (se) to (sci);
	\draw[->] (se) to (eng);
	\draw[->] (se) to (prod);
	\draw[->] (se) to (ec);
	\draw[->] (se) to (man);
\end{tikz*}

	}
	
	\section[Наука]{Научная дисциплина ПИ}
	
	\subsection{Определение дисциплины}
	
	\frame{
		\frametitle{Научная дисциплина ПИ}
		
		\textbf{Цель программной инженерии как науки:} 
		применение знаний, полученных из~фундаментальных наук, для~построения сложных программных продуктов.
		
		\vspace{1ex}
		\textbf{Построение программного продукта} — анализ автоматизируемой предметной области 
		и~производство кода для~выполнения на компьютере.
		
		\vspace{1ex}
		\textbf{Фундаментальные науки}, используемые в~ПИ — это:
		\begin{itemize}
			\item теория алгоритмов;
			\item математическая логика;
			\item теория управления;
			\item теория доказательств;
			\item теория множеств.
		\end{itemize}
	}

	\frame{
		\frametitle{Составляющие ПИ как науки}
		
		ПИ как наука содержит в себе:
		\begin{enumerate}
			\item
			основные понятия и объекты;
			\item
			теорию программирования;
			\item 
			методы управления изготовлением ПО;
			\item
			средства и инструменты процессов изготовления.
		\end{enumerate}
	}

	\subsection{Основные понятия ПИ}
	
	\frame{	
		\frametitle{Основные понятия ПИ}
		
		\textbf{Понятия программной инженерии:}
		\begin{itemize}
			\item
			данные и структуры данных;
			\item
			функции и композиции над данными;
			\item
			базовые объекты (модули, компоненты, каркасы, контейнеры и т.\,п.);
			\item
			целевые объекты (программное обеспечение, программная система, семейство систем, программный проект и т.\,п.).
		\end{itemize}
		
		Целевые объекты изготавливаются из базовых при помощи инженерных методов, включая управление сроками и затратами на производство.
		
		\vspace{2ex}
		\begin{tikz*}[%
	every node/.style={rectangle,align=center,minimum height=3em,inner xsep=0.5em,minimum width=15em},
	label/.style={font=\footnotesize\itshape,minimum height=0pt,minimum width=0pt}
]
	\node(nonfunc) {\bfseries Нефункциональное требование};
	\node(func) [right=6em of nonfunc]{\bfseries Функциональное требование};
	\node(confid) [below=of nonfunc] {Защита конфиденциальных \\ данных};
	\node(auth) [below=of func] {Система авторизации};
	\node(mem) [below=of confid] {Ограничение на \\ занимаемую память};
	\node(del) [below=of auth] {Периодическое удаление \\ лишних данных};
	\node(recover) [below=of mem] {Отказоустойчивость};
	\node(copy) [below=of del] {Система резервных \\ копий данных};
	
	\draw[->] (confid) -- node[below,label]{уточнение} (auth);
	\draw[->] (mem) -- node[below,label]{уточнение} (del);
	\draw[->] (recover) -- node[below,label]{уточнение} (copy);
\end{tikz*}

	}

	\frame{	
		\frametitle{Определение целевых объектов ПИ}

		\begin{Definition}
			\textbf{Программная (прикладная) система} (англ. \emph{application}) — комплекс интегрированных программ и средств, 
			реализующих набор взаимосвязанных функций некоторой предметной области в заданной среде.
		\end{Definition}

		\vspace{.5ex}
		\begin{Definition}
			\textbf{Программное обеспечение} (англ. \emph{software product}) — совокупность программных инструментов, 
			реализующих определенную функцию компьютерной системы.
		\end{Definition}

		\vspace{.5ex}
		\begin{Definition}
			\textbf{Семейство систем} (англ. \emph{software product family}) — совокупность программных систем с~общими и переменными характеристиками, 
			удовлетворяющих заданные потребности предметной области.
		\end{Definition}
	}

	\subsection{Теория программирования}
	
	\frame{	
		\frametitle{Теория программирования}

		\begin{Definition}
			\textbf{Теория программирования} — совокупность методов, языков и~средств описания и~проектирования целевых объектов ПИ, 
			а~также методы их~доказательства, верификации и~тестирования.
		\end{Definition}

		\vspace{1ex}
		\textbf{Составляющие теории программирования:}
		\begin{itemize}
			\item
			методы программирования (теоретические и прикладные);
			\item
			методы проверки правильности;
			\item
			методы оценки промежуточных результатов проектирования и конечного продукта относительно показателей 
			(надежность, качество, точность, продуктивность и~т.\,д.);
			\item
			методы управления и контроля разработки.
		\end{itemize}
	}

	\section[Инженерия]{Инженерная дисциплина ПИ}
	
	\subsection{Определение дисциплины}
	
	\frame{
		\frametitle{Программная инженерия как инженерная дисциплина}

		\textbf{Цель ПИ как инженерной дисциплины:} производство программного обеспечения.

		\vspace{1ex}
		Основу инженерии ПО составляют:
		\begin{itemize}
			\item
			ядро знаний SWEBOK — теоретическая основа и формальные определения методов и~средств разработки;
			\item
			базовый процесс ПИ — собственно процессы жизненного цикла ПО;
			\item
			инфраструктура — среда разработки;
			\item
			стандарты ПИ — регламентированные правила построения промежуточных артефактов в~процессах ЖЦ;
			\item
			менеджмент проекта (PMBOK, ядро знаний по~управлению промышленными проектами) — набор стандартов, 
			принципов и~методов планирования и~контроля работ в~проекте;
			\item
			прикладные средства и~инструменты разработки программных продуктов.
		\end{itemize}
	}
	
	\subsection{Инфраструктура программной инженерии}

	\frame{	
		\frametitle{Инфраструктура программной инженерии}

		\begin{Definition}
			\textbf{Инфраструктура} — набор технических, технологических, программных и людских ресурсов организации-разработчика, 
			необходимых для исполнения подпроцессов базового процесса программной инженерии.
		\end{Definition}

		\vspace{2ex}
		\textbf{Технические ресурсы:} компьютеры и компьютерная техника, серверы и т.\,п.

		\vspace{1ex}
		\textbf{Программные ресурсы:} общесистемное ПО, среда разработки, наработки организации-разработчика, информационное обеспечение.

		\vspace{1ex}
		\textbf{Технологические ресурсы:} методики, процедуры, правила, рекомендации, документы, регламентирующие процесс разработки.

		\vspace{1ex}
		\textbf{Людские ресурсы:} группы разработчиков и менеджеров разработки, тестирования, обеспечения качества, оценки риска, конфигурации и т.\,д.
	}
	
	\subsection{Стандарты программной инженерии}

	\frame{
		\frametitle{Стандарты программной инженерии}

		\textbf{Основные стандарты}, касающиеся программной инженерии:
		\begin{itemize}
			\item
			ISO/IEC 12207 «Процессы жизненного цикла программного обеспечения»;
			\item
			ISO/IEC 14598 «Оценивание программного продукта»;
			\item
			ISO 15939 «Процесс измерения»;
			\item
			ISO/IEC 15504 «Оценивание процессов жизненного цикла ПО»;
			\item
			ISO 9001 «Системы управления качеством»;
			\item
			ISO/IEC TR 9126 «Программная инженерия. Качество продукта».
		\end{itemize}
	}
	
	\section[Производство]{Производственная дисциплина ПИ}
	
	\subsection{Определение дисциплины}
	
	\frame{
		\frametitle{Программная инженерия как производственная дисциплина}
		
		\textbf{Цель ПИ как производственной дисциплины:} 
		изготовление программных продуктов с~использованием доступных разработанных 
		готовых программ и~информационных ресурсов сети Интернет.
		
		\vspace{1ex}
		Программные продукты строятся из~\textbf{компонентов повторного использования} (КПИ, англ.~\emph{reuse}), 
		которые хранятся в~свободном доступе в~\textbf{репозиториях программного обеспечения}.
		
		\vspace{3ex}
		\begin{tikz*}[%
	every node/.style={rectangle,draw,align=center},
	label/.style={draw=none,font=\small\itshape}
]
	\node(reuse1) {КПИ$_1$};
	\node(reuse2) [right=0.75em of reuse1] {КПИ$_2$};
	\node(reuse-dots) [right=0.75em of reuse2,draw=none] {\dots};
	\node(reusen) [right=0.75em of reuse-dots] {КПИ$_n$};
	\node(repo-label) [label,below=2.5em of $(reuse1.south west)!0.5!(reusen.south east)$] {репозиторий};
	
	\node(model) [right=5em of reusen] {Модель ПрО};
	\node(conf) at (model.center |- repo-label.center) {Конфигурация};
	
	\begin{pgfonlayer}{background}
		\node(repo) [fit=(reuse1) (reuse2) (reusen) (repo-label),rectangle,draw,fill=blue!20,minimum height=6em] {};
	\end{pgfonlayer}

	\node(prod) [minimum height=6em,right=15em of repo] {Производство ПП};
	
	\draw[->] (repo.north east |- model.west) to (model.west);
	\draw[->] (model.east) to (prod.north west |- model.east);
	\draw[<-] (repo.south east |- conf.west) to (conf.west);
	\draw[<-] (conf.east) to (prod.south west |- conf.east);
\end{tikz*}

	}
	
	\subsection{Применение КПИ}

	\frame{
		\frametitle{Применение КПИ}

		\textbf{Инженерные подходы} к применению КПИ:
		\begin{itemize}
			\item
			инженерия КПИ (англ. \emph{reuse engineering}): проектирование системы снизу вверх;
			\item
			инженерия приложений (англ. \emph{application engineering}): проектирование системы сверху вниз;
			\item
			инженерия предметной области (англ. \emph{domain engineering}): одновременное проектирование семейства программных продуктов; 
			напоминает процесс конвейерной сборки.
		\end{itemize}
	}

	\section[Управление]{ПИ как дисциплина управления}
	
	\subsection{Определение дисциплины}
	
	\frame{
		\frametitle{Программная инженерия как дисциплина управления}

		\textbf{Цель ПИ как дисциплины управления:} автоматизация и~оптимизация управления процессами разработки программного продукта.

		\vspace{1ex}
		Теоретическая основа дисциплины управления — теория управления сложными системами, разработанная в~1970-х годах В.\,М. Глушковым.
	}
	
	\subsection{Стандарт PMBOK}
	
	\frame{
		\frametitle{Стандарт PMBOK}

		Стандарты управления собраны в ядре знаний PMBOK (2000 г., Институт управления проектами США) и стандарте IEEE Std.1490 и включают:

		\begin{itemize}
			\item
			управление \textbf{содержанием проекта}: процессы, необходимые для выполнения работ по~проекту 
			и~его планирования с~расщеплением работ на~более простые для~упрощения процесса управления;
			\item
			управление \textbf{качеством}: процессы, связанные с~обеспечением качества согласно заданным условиям и~контролем качества конечного продукта;
			\item
			управление \textbf{человеческими ресурсами}: организация и~распределение работ между~исполнителями согласно их~квалификации.
		\end{itemize}
	}
	
	\section[Экономика]{Экономическая дисциплина ПИ}
	
	\subsection{Определение дисциплины}
	
	\frame{	
		\frametitle{Экономическая дисциплина программной инженерии}

		\textbf{Цель экономики программной инженерии:}
		\begin{itemize}
			\item
			оценка ценовых, временных и экспертных \textbf{показателей} для составления контрактов 
			на~создание программного продукта, принятия проектных решений, разработки архитектуры и~т.\,п.;
			\item
			определение \textbf{рисков} проектирования при заданных ресурсах.
		\end{itemize}

		\vspace{1ex}
		Для оценки затрат на производство программных продуктов используются математические модели: 
		COCOMO, ANGEL, SLIM и~т.\,д.
	}

	\section{Заключение}
	
	\subsection{Выводы}

	\frame{
		\frametitle{Выводы}
		
		\begin{enumerate}
			\item
			Основными составляющими программной инженерии являются ее~научное, инженерное и~производственное направления, 
			а также дисциплины управления и экономики.

			\vspace{0.5ex}
			\item
			Процессы программной инженерии описаны в базах знаний SWEBOK и~PMBOK и~в~стандартах ISO.

			\vspace{0.5ex}
			\item
			Все дисциплины ПИ связаны между собой процессами жизненного цикла программного обеспечения, 
			методами проектирования и управления программными проектами.
		\end{enumerate}
	}
	
	\subsection{Материалы}

	\frame{
		\frametitle{Материалы}
		
		\begin{thebibliography}{9}
			\bibitem{1}
			Лавріщева К.\,М.
			\newblock Програмна інженерія (підручник). 
			\newblock {\footnotesize К., 2008. — 319~с.}
			
			\bibitem{2}
			Pfleeger S.\,L.
			\newblock Software Engineering. Theory and Practice. 
			\newblock {\footnotesize Prentice Hall, NJ, 1998. — 576~p.}
			
			\bibitem{3}
			Иан Соммервил.
			\newblock Инженерия программного обеспечения. 
			\newblock {\footnotesize 6-е издание. М.; СПб, 2002. — 623~с.}
		\end{thebibliography}
	}
	
	\frame{
		\frametitle{}
		
		\begin{center}
			\Huge Спасибо за внимание!
		\end{center}
	}
\end{document}
