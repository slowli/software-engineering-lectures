\begin{tikz*}[%
	every node/.style={rectangle,draw,align=center,minimum height=3em},
	server/.style={rectangle split,rectangle split parts=2,rectangle split part align=center},
	every label/.style={draw=none,minimum height=0pt,font=\bfseries}
]
	\node(server1) [server,label=above:Сервер 1] {
		Реализация\strut{}
		\nodepart{two}
		Интерфейс\strut{}
	};
	\node(server2) [server,label=above:Сервер 2,right=of server1] {
		Реализация\strut{}
		\nodepart{two}
		Интерфейс\strut{}
	};
	\node(server-dots) [draw=none,right=of server2] {$\cdots$};
	\node(server3) [server,label=above:Сервер $n$,right=of server-dots] {
		Реализация\strut{}
		\nodepart{two}
		Интерфейс\strut{}
	};
	
	\node(middleware) [below=of $ (server1.south west)!0.5!(server3.south east) $,minimum width=30em,minimum height=3.5em] {};
	\node [above=0pt of middleware.south,draw=none,minimum height=0pt] {Промежуточный слой \engterm{middleware}};
	
	\node(_tmp) [coordinate,below=of middleware.south] {};
	\node(client2) at (server2.south |- _tmp) [anchor=north] {\textbf{Клиент 2}};
	\node(client-dots) at (server-dots.south |- _tmp) [draw=none,anchor=north] {$\cdots$};
	\node(client1) at (server1.south |- _tmp) [anchor=north] {\textbf{Клиент 1}};
	\node(client3) at (server3.south |- _tmp) [anchor=north] {\textbf{Клиент $m$}};
	
	\draw[<->] (server1.south) to (middleware.north -| server1.south);
	\draw[<->] (server2.south) to (middleware.north -| server2.south);
	\draw[<->] (server3.south) to (middleware.north -| server3.south);
	\draw[<->] (client1.north) to (middleware.south -| client1.north);
	\draw[<->] (client2.north) to (middleware.south -| client2.north);
	\draw[<->] (client3.north) to (middleware.south -| client3.north);
	
	\draw[<->,dotted,bend right] (server1.south |- middleware.north) to node(server-link)[coordinate]{} (server2.south |- middleware.north);
	\node<2> [note={(server-link)},below=1em of server-link,anchor=north west] {%
		Сервера могут обмениваться \\ данными между собой \\ прозрачно для клиентов.
	};
\end{tikz*}
