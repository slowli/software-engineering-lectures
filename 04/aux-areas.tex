\documentclass{a4beamer}
%% Lectures - common definitions

\usextensions{tikz}
\usetikzlibrary{shapes.multipart,shapes.callouts,shapes.geometric}
\input{fix-callouts.inc} % Fixes absolute positioning of rectangle callouts

\newif\ifbigpages \bigpagesfalse
\ifdim\paperwidth >20cm
	\bigpagestrue
\fi

\tikzset{%
	note/.style={rectangle callout,draw=none,callout pointer width=1em,%
		align=flush left,font=\footnotesize,inner sep=0.5em,%
		fill=blue!15,fill opacity=0.95,text opacity=1.0,callout absolute pointer=#1},
	node distance=2em and 2.75em
}
\ifbigpages
	% Scale all arrow tips by the factor of 2.5
	\let\old@pgf@arrow@call=\pgf@arrow@call
	\def\pgf@arrow@call#1{%
		\@tempdima=\pgflinewidth%
		\pgfsetlinewidth{2.5\pgflinewidth}%
		\old@pgf@arrow@call{#1}%
		\pgfsetlinewidth{\@tempdima}%
	}
	\def\pgfarrowsleftextend#1{\pgfmathsetlength{\pgf@xa}{1.5*#1}}
	\def\pgfarrowsrightextend#1{\pgfmathsetlength{\pgf@xb}{1.5*#1}}
\fi

%% Explanation for a figure
\def\figureexpl#1{%
	{\linespread{1.0}\footnotesize\color{gray}{#1}\par}
}
%% English term
\def\engterm#1{(англ. \textit{#1})}
%% Term with explanation below (to be used in diagrams)
\def\termwithexpl#1#2{#1\strut{}\\\small\color{gray}(\textit{#2})\strut{}}
%% External link
\def\extlink#1#2{\href{#1}{\color[rgb]{0.7,0.7,1.0}\dashbar{#2}}}
%% Internal link
\def\inlink#1#2{\hyperlink{#1}{\color[rgb]{0.7,0.7,1.0}\dashbar{#2}}}
%% Explanation for a list item
\def\itemexpl#1{\begingroup\linespread{1.0}\small\vspace{0.75ex}#1\par\endgroup}



\usepackage{array}
\usepackage{multicol}
\usetikzlibrary{decorations.pathreplacing}


\lecturetitle{Программная инженерия. Лекция №4 — SWEBOK: организационные области знаний.}
\title[SWEBOK. Огранизационные области знаний]{SWEBOK. Огранизационные области знаний}
\author{Алексей Островский}
\institute{\small{Физико-технический учебно-научный центр НАН Украины}\vspace{2ex}}
\date{10 октября 2014 г.}

\begin{document}
	\frame{\titlepage}
	
	\frame{
		\frametitle{Организационные области знаний SWEBOK}
		
		\def\areamember#1{\small\color{gray}#1\strut{}}
\begin{tikz*}[%
	area/.style={rectangle split,rectangle split parts=2,draw,align=center}
]
	\node(areas) at (0, 0) [rectangle,draw,minimum width=35em] {\large Организационные области знаний SWEBOK\strut{}};
	

	\node(tools) [area,below=1em of areas] {
		\textbf{Инструменты и методы}
		\nodepart{two}
		\areamember{инструменты} \\ 
		\areamember{методы} \\
		\areamember{} \\
		\areamember{}
	};
	\node(process) [area,left=1em of tools.north west,anchor=north east] {
		\textbf{Процесс инженерии}
		\nodepart{two}
		\areamember{имплементация} \\ 
		\areamember{определение} \\ 
		\areamember{оценка} \\ 
		\areamember{измерение}
	};
	\node(quality) [area,right=1em of tools.north east,anchor=north west] {
		\textbf{Инженерия качества}
		\nodepart{two}
		\areamember{концепция} \\ 
		\areamember{характеристики} \\ 
		\areamember{QA} \\ 
		\areamember{верификация и валидация}
	};

	\node(_tmp) at (areas.north -| process.east) [coordinate]{};
	\node(conf) [area,above=1em of _tmp,anchor=south] {
		\textbf{Управление конфигурацией}
		\nodepart{two}
		\areamember{управление процессом} \\ 
		\areamember{идентификация} \\ 
		\areamember{контроль} \\ 
		\areamember{учет статуса} \\ 
		\areamember{аудит} \\
		\areamember{выпуски и доставка}
	};
	\node(_tmp) at (areas.north -| quality.west) [coordinate]{};
	\node(eng) [area,above=1em of _tmp,anchor=south] {
		\textbf{Управление инженерией}
		\nodepart{two}
		\areamember{планирование} \\ 
		\areamember{координация} \\
		\areamember{мониторинг} \\
		\areamember{контроль} \\ 
		\areamember{измерение} \\
		\areamember{отчетность}
	};
	
	\draw[->] (areas.north -| conf.south) to (conf.south);
	\draw[->] (areas.north -| eng.south) to (eng.south);
	\draw[->] (areas.south -| process.north) to (process.north);
	\draw[->] (areas.south -| tools.north) to (tools.north);
	\draw[->] (areas.south -| quality.north) to (quality.north);
\end{tikz*}

	}
	
	\section[Конфигурация]{Управление конфигурацией}
	
	\subsection{Определение}
	
	\frame{
		\frametitle{Управление конфигурацией ПО}

		\begin{Definition}
			\textbf{Конфигурация} — совокупность функциональных и/или физических характеристик оборудования 
			(\emph{hardware}), прошивок (\emph{firmware}) и ПО (\emph{software}), 
			определенных в~технической документации и реализованных программным продуктом.
		\end{Definition}

		\vspace{1ex}
		\begin{Definition}
			\textbf{Управление конфигурацией} \engterm{software control management} — процесс идентификации конфигурации 
			программной системы в заданные моменты времени с~целью:
			\begin{itemize}
				\item
				систематического контроля изменений конфигурации;
				\item
				поддержки целостности (\emph{integrity}) и отслеживаемости (\emph{traceability}) конфигурации 
				на~протяжении жизненного цикла ПО.
			\end{itemize}
		\end{Definition}
	}

	\frame{
		\frametitle{Составляющие управления конфигурацией}
		
		\begin{figure}
			{\small\linespread{1.0}
\begin{tikz*}[%
	every node/.style={rectangle,draw,align=center}
]
	\node(control) at (0, 0) [minimum height=6em] {
		\textbf{Контроль} \\ \textbf{конфигурации} \\
		\small\color{gray} (SC control)
	};
	\node(status) [right=of control,minimum height=6em] {
		\textbf{Учет статуса} \\ \textbf{конфигурации} \\ 
		\small\color{gray} (SC status \\
		\small\color{gray} accounting)
	};
	\node(release) [right=of status,minimum height=6em] {
		\textbf{Выпуски и} \\ \textbf{доставка} \\
		\small\color{gray} (release management \\
		\small\color{gray} and delivery)
	};
	\node(audit) [right=of release,minimum height=6em] {
		\textbf{Аудит} \\ \textbf{конфигурации} \\ 
		\small\color{gray} (SC auditing)
	};
	
	\node(midx) at ($ (control)!0.5!(audit) $) [draw=none] {};
	
	\node(ident) [below=4em of midx,minimum width=35em] {
		\textbf{Идентификация конфигурации} \\
		\small\color{gray} (SC identification)
	};
	
	\draw[->] (ident.north -| control.south) to (control.south);
	\draw[->] (ident.north -| status.south) to (status.south);
	\draw[->] (ident.north -| release.south) to (release.south);
	\draw[->] (ident.north -| audit.south) to (audit.south);
	
	\node(mgmt) [below=of ident,draw=none] {
		\textbf{Управление процессом конфигурации} \\
		\small\color{gray}(management of the SCM process)
	};
	
	\node[dashed,inner xsep=1em,inner ysep=1em,fit=(control.north west) (audit.north east) (mgmt.south)] {};
\end{tikz*}
}
			\caption{\textbf{Сокращения:} SC = software configuration; SCM = software configuration management.}
		\end{figure}
	}
	
	\subsection{Управление процессом конфигурации}
	
	\frame{
		\frametitle{Управление процессом конфигурации}

		\begin{Definition}
			\textbf{Управление процессом конфигурации} \engterm{management of the SCM process} — деятельность 
			по контролю эволюции и целостности программного продукта следующими методами:
			\begin{itemize}
				\item
				идентификация элементов ПО;
				\item
				управление и контроль изменений, вносимых в продукт;
				\item
				верификация, сохранение и генерация отчетов по конфигурации ПО.
			\end{itemize}
		\end{Definition}

		\vspace{1ex}
		\textbf{Цель:} составление \emph{плана управления} процессом конфигурации, который включает:
		\begin{itemize}
			\item
			описание \emph{деятельности} по конфигурации (идентификация, контроль, аудит, …);
			\item
			\emph{расписание} процессов конфигурации;
			\item
			используемые \emph{ресурсы} (инструменты, ответственные люди, …).
		\end{itemize}
	}
	
	\subsection{Идентификация конфигурации}
	
	\frame{
		\frametitle{Идентификация конфигурации ПО}

		\textbf{Этапы} идентификации конфигурации:
		\begin{enumerate}
			\item
			идентификация контролируемых элементов;
			\item
			создание схем идентификации для объектов и их версий;
			\item
			определение инструментов и методов для получения и управления контролируемыми элементами.
		\end{enumerate}

		\vspace{1ex}
		\textbf{Элементы} конфигурации: 
		\vspace{-0.5ex}
		\begin{multicols}{2}
			\begin{itemize}
				\item исполняемый и исходный код;
				\item планы;
				\item спецификации и проектная документация;
				\item система тестирования;
				\item программные инструменты; 
				\item сторонние библиотеки;
				\item справочная документация.
			\end{itemize}
		\end{multicols}

		\textbf{Базис} \engterm{baseline} — набор элементов конфигурации, формально описанных и зафиксированных 
		в конкретный момент времени.
	}
	
	\subsection{Контроль конфигурации}
	
	\frame{
		\frametitle{Контроль конфигурации ПО}

		\begin{figure}
			{\small\linespread{1.0}\small%
%\def\hilight{\color{red}\bfseries}%
\begin{tikz*}[xscale=1.25,
	every node/.style={rectangle,draw,align=center,minimum height=3em,minimum width=9.5em},
	condition/.style={diamond,shape aspect=2,minimum width=10.7em,minimum height=5.4em},
	edge/.style={minimum height=0pt,minimum width=0pt,draw=none,font=\footnotesize\itshape},
]
	\node(need) at (0, 0) [ellipse,minimum height=4.25em] {
		необходимо \\ изменение
	};
	\node(ident) [below=of need] {
		идентификация \\ элемента к.
	};
	\node(request) [below=of ident] {
		запрос изменения \\ элемента к.
	};
	\node(compl) [condition,below=of request] {
		оценка полноты
	};
	\node(enrich) [left=of request] {
		дополнение
	};
	\node(content) [condition,right=5em of compl] {
		оценка сути
	};
	\node(inform) [below=of content] {
		информирование \\ заказчика
	};
	\node(schedule) [above=of content] {
		расписание и \\ проект
	};
	\node(test) [above=of schedule] {
		составление \\ тестов
	};
	\node(impl) [above=of test] {
		внесение \\ изменения
	};
	
	\draw[->] (need) to (ident);
	\draw[->] (ident) to (request);
	\draw[->] (request) to (compl);
	\draw[->] (compl) -| (enrich) node[edge,below,pos=0.3]{неполное};
	\draw[->] (enrich) |- ($ (ident)!0.5!(request) $);
	\draw[->] (compl) -- node[edge,below]{полное} (content);
	\draw[->] (content) -- node[edge,right]{отказ} (inform);
	\draw[->] (content) -- node[edge,right]{утверждение} (schedule);
	\draw[->] (schedule) to (test);
	\draw[->] (test) to (impl);
\end{tikz*}
}
			\caption{%
				\textbf{Контроль конфигурации ПО} — деятельность по координации и оценке реализованных изменений 
				элементов конфигурации, а также их утверждению или отбрасыванию.}
		\end{figure}
	}
	
	\subsection{Учет статуса и аудит}

	\frame{
		\frametitle{Учет статуса и аудит конфигурации}

		\begin{Definition}
			\textbf{Учет статуса конфигурации} — это запоминание информации и~составление отчетов 
			для~определения степени изменения конфигурации и~правильности вносимых~в~ПО изменений.
		\end{Definition}

		\vspace{1ex}
		\begin{Definition}
			\textbf{Аудит конфигурации} — деятельность для оценки соответствия ПО и процессов разработки 
			регламентированным стандартам, инструкциям, планам и процедурам:
			\begin{itemize}
				\item
				\emph{функциональный} аудит (проверка элемента ПО на соответствие спецификации);
				\item
				\emph{физический} аудит (проверка соответствия элемента проектной и справочной документации).
			\end{itemize}
		\end{Definition}
	}
	
	\subsection{Управление выпусками и доставкой}

	\frame{
		\frametitle{Управление выпусками и доставкой}

		{\begin{tikz*}[%
	every node/.style={rectangle,draw,align=center,minimum height=3em,minimum width=8.5em},
	edge/.style={draw=none,minimum height=0pt,minimum width=0pt,font=\small\itshape},
	hilight/.style={font=\only<#1>{\color{red}}},
	hilight edge/.style={font=\footnotesize\itshape\only<#1>{\color{red}}}
]
	\node(elem) {элементы \\ конфигурации};
	\node(lib) [right=7.5em of elem,hilight=2] {библиотека};
	\node(user) [above right=of lib] {конечный \\ пользователь};
	\node(test) [below right=of lib] {тестирование};
	
	\draw[->] (elem) -- node[edge,above,hilight edge=1]{построение; \\ конф. данные} (lib);
	\draw[->] (lib) -- node[edge,below right,hilight edge=3]{публичный релиз} (user);
	\draw[->] (lib) -- node[edge,above right,hilight edge=3]{технический релиз} (test);
\end{tikz*}

		\\[1ex]}
		
		\begin{overlayarea}{\textwidth}{0.4\textheight}
			\only<1>{
				\begin{Definition}
					\textbf{Построение ПО} — объединение корректных версий элементов конфигурации 
					с~использованием конфигурационных данных в исполняемую программу для доставки заказчику 
					или другому потребителю (напр., отделу тестирования).
				\end{Definition}
			}
			\only<2>{
				\begin{Definition}
					\textbf{Программная библиотека} — контролируемая коллекция программ и соответствующей документации, 
					предназначенная для упрощения разработки, использования и~сопровождения ПО.
				\end{Definition}
			}
			\only<3>{
				\begin{Definition}
					\textbf{Выпуск (релиз) ПО} — деятельность по определению, подготовке и доставке элементов программного продукта: 
					исполняемой программы, документации, конфигурационных данных и т.\,п.
				\end{Definition}
			}
		\end{overlayarea}
	}
	
	\section[Инженерия]{Управление инженерией}
	
	\subsection{Определение}

	\frame{
		\frametitle{Управление инженерией ПО}

		\begin{Definition}
			\textbf{Управление инженерией ПО} \engterm{software engineering management} — управление работами команды разработки~ПО 
			в~процессе выполнения плана проекта, определение критериев эффективности работы этой команды 
			и~оценка процессов и~продуктов проекта с~использованием общих методов планирования и~контроля работ.
		\end{Definition}

		\vspace{1ex}
		\textbf{Составляющие:} планирование, координация, мониторинг, контроль, измерение и~отчетность.

		\vspace{1ex}
		\textbf{Уровни управления:}
		\begin{itemize}
			\item организационное и инфраструктурное управление; 
			\item управление проектом; 
			\item планирование и контроль измерений.
		\end{itemize}
	}
	
	\subsection{Организационное управление}

	\frame{
		\frametitle{Организационное управление}

		\textbf{Составляющие} организационного управления:
		\begin{itemize}
			\item
			планирование и составление графика работ, связанных с~проектом;
			\item
			подбор и управление персоналом;
			\item
			контроль выполнения проекта и оценка качества работ в соответствии со~стандартами и~договором с~заказчиком.
		\end{itemize}

		\vspace{1ex}
		\textbf{Объекты} управления:
		\begin{itemize}
			\item
			персонал (обучение, мотивация, карьерный рост, …);
			\item
			коммуникации между сотрудниками (встречи, презентации, …);
			\item
			портфель \engterm{portfolio} (организация повторного использования кода).
		\end{itemize}
	}
	
	\subsection{Управление проектом}

	\frame{
		\frametitle{Управление программным проектом}

		\begin{enumerate}
			\item
			\textbf{Инициирование} проекта и определение его рамок.

			\itemexpl{\textbf{Составляющие:} определение и согласование требований; анализ выполнимости (технической, операционной, 
			финансовой, социальной); пересмотр требований.}

			\item
			\textbf{Планирование} проекта.
			
			\itemexpl{\textbf{Составляющие:} планирование процессов ЖЦ; определение промежуточных артефактов; оценка затрат; 
			распределение ресурсов; управление риском; управление качеством; мониторинг проекта.}

			\item
			\textbf{Реализация} проекта.
			
			\itemexpl{\textbf{Составляющие:} реализация планов; управление контрактами с поставщиками; имплементация измерительных процессов; 
			мониторинг и контроль; отчетность.}

			\item
			\textbf{Рецензирование и оценка} проекта.

			\itemexpl{\textbf{Составляющие:} определение соответствие проекта требованиям; оценка эффективности выполненных работ.}

			\item
			\textbf{Завершение} проекта.

			\itemexpl{\textbf{Составляющие:} определение степени завершенности проекта; деятельность по завершению (напр.,~\emph{post-mortem}).}
		\end{enumerate}
	}

	\subsection{Инженерия измерения}
	
	\frame{
		\frametitle{Инженерия измерения ПО}

		\textbf{Этапы инженерии измерения}
		
		\vspace{2ex}
		{\linespread{1.0}
		\begin{tikz*}[every node/.style={rectangle,align=center,minimum height=3em},node distance=2em and 2.5em]
			\node(def) {определение \\ метрик};
			\node(org) [right=of def] {организация}; 
			\node(plan) [right=of org] {планирование}; 
			\node(measure) [right=of plan] {измерение}; 
			\node(assess) [right=of measure] {оценка};
			
			\draw[->] (def) to (org);
			\draw[->] (org) to (plan);
			\draw[->] (plan) to (measure);
			\draw[->] (measure) to (assess);
		\end{tikz*}\strut{}}
		
		\textbf{Цели} инженерии измерения:
		\begin{itemize}
			\item усовершенствование процессов управления проектом;
			\item оценка и регулировка временных затрат и стоимости ПО;
			\item определение категорий рисков и отслеживание факторов для расчета вероятностей их~возникновения;
			\item проверка заданных в требованиях показателей качества продуктов и~проекта в~целом.
		\end{itemize}
	}

	\section[Процесс]{Процесс программной инженерии}
	
	\subsection{Основные определения}
	
	\frame{
		\frametitle{Процесс программной инженерии}

		\begin{Definition}
			\textbf{Процесс программной инженерии} \engterm{software engineering process} — деятельность по~определению, 
			имплементации, оценке, измерению, управлению, изменению и~совершенствованию процессов жизненного цикла ПО.
		\end{Definition}

		\vspace{1ex}
		\textbf{Составляющие} области знаний:
		\begin{itemize}
			\item
			имплементация и изменение процессов ЖЦ;

			\itemexpl{(инфраструктура, деятельность, модели и практические соображения, касающиеся имплементации и модификации процессов жизненного цикла)}

			\item определение процессов;
			\item оценка процессов;
			\item измерение процессов ЖЦ и программного продукта.
		\end{itemize}
	}
	
	\subsection{Определение процессов}

	\frame{
		\frametitle{Определение процессов ЖЦ}

		\textbf{Цель} определения процессов:
		\begin{itemize}
			\item повышение качества программного продукта;
			\item облегчение взаимопонимания и коммуникации между разработчиками;
			\item поддержка совершенствования процессов и управления ними;
			\item автоматизация руководства процессами.
		\end{itemize}

		\vspace{1ex}
		\textbf{Инструменты} определения:
		\begin{itemize}
			\item модель жизненного цикла (каскадная, спиральная, итеративная, …);
			\item стандарты ЖЦ ПО (ISO/IEC 12207, ISO/IEC 15504, IEEE 1074, IEEE 1219, …);
			\item нотации представления процессов (диаграммы потоков данных, сети Петри, \\ IDEF0, …);
			\item инструменты автоматизации.
		\end{itemize}
	}

	\subsection{Оценка процессов}
	
	\frame{
		\frametitle{Оценка процессов ЖЦ}

		\linespread{1.0}%
\begin{tikz*}[%
	xscale=1.25,
	every node/.style={rectangle,align=center,minimum height=3em}
]
	\node(features) {\textbf{Оцениваемые характеристики процессов}};
	\node(plan) [below=of features] {планируемость};
	\node(control) [left=of plan] {управляемость};
	\node(predict) [below=of control] {предсказуемость};
	\node(product) [right=of plan] {результативность};
	\node(measure) [below=of product] {измеримость};
	
	\draw[-] (features) to (plan);
	\draw[-] (features) to (control);
	\draw[-] (features) to (predict);
	\draw[-] (features) to (product);
	\draw[-] (features) to (measure);
	
	\draw[decorate,decoration={brace,amplitude=6pt}] (measure.south east) -- node[below]{зрелость} (predict.south west);
\end{tikz*}

		\vspace{1ex}
		\textbf{Модели} оценки процессов: 
		\begin{itemize}
			\item SW-CMM (software capability maturity model); 
			\item CMMI (capability maturity model integration); 
			\item Bootstrap; 
			\item стандарты ISO/IEC 15504, ISO 9001.
		\end{itemize}
	}

	\subsection{Измерение процессов и продукта}
	
	\frame{
		\frametitle{Измерение процессов и продукта}

		\begin{Definition}
			\textbf{Измерение процессов жизненного цикла} — сбор, анализ  и интерпретация количественных характеристик процессов 
			с целью обнаружить их сильные и слабые стороны, а также оценить качество имплементации и/или внесенных изменений.
		\end{Definition}

		\vspace{1ex}
		\textbf{Примеры характеристик:}
		\begin{itemize}
			\item качество продукта (напр., число дефектов на 1 строку кода);
			\item простота сопровождения (затраты на проведение определенного изменения);
			\item продуктивность (напр., в строках кода на человеко-месяц);
			\item время разработки;
			\item удовлетворенность пользователей (напр., измеренная анкетированием).
		\end{itemize}
	}
	
	\section{Методы и инструменты}
	
	\subsection{Определения}
	
	\frame{
		\frametitle{Методы и инструменты ПИ}

		\begin{Definition}
			\textbf{Инструменты разработки ПО} \engterm{software development tools} — это компьютерные инструменты, 
			предназначенные для упрощения процессов ЖЦ путем автоматизации формализованных повторяющихся действий.
		\end{Definition}

		\vspace{1ex}
		\begin{Definition}
			\textbf{Методы ПИ} \engterm{software engineering methods} упорядочивают деятельность, связанную с~программной инженерией, 
			чтобы систематизировать действия по~разработке ПО и~увеличить их~продуктивность.
		\end{Definition}
	}

	\subsection{Методы}
	
	\frame{
		\frametitle{Методы ПИ}
		
		\begin{itemize}
			\item
			\textbf{Эвристические методы} — методы, основанные на неформальном подходе к ПИ.

			\textbf{Примеры:} структурные, объектно-ориентированные, информационные (\emph{data-oriented}).

			\item
			\textbf{Формальные} методы — основанные на математических моделях.

			\textbf{Примеры:} языки спецификации; методы уточнения спецификации (\emph{specification refinement}) — 
			ее приближение к форме конечного продукта; методы доказательства и~верификации.

			\item
			Методы \textbf{прототипирования} — техники создания прототипов ПО.
		\end{itemize}
	}
	
	\subsection{Инструменты}
	
	\frame{
		\frametitle{Инструменты ПИ}

		\begin{itemize}
			\item Инструменты работы с \textbf{требованиями}: моделирование и отслеживание (\emph{traceability}) требований;
			\item инструменты \textbf{проектирования};
			\item инструменты \textbf{конструирования}: редакторы программ; компиляторы и генераторы кода; интерпретаторы;  отладчики;
			\item инструменты \textbf{тестирования}: генераторы тестов; системы тестирования (\emph{test execution framework}); 
			инструменты оценки тестов; инструменты управления тестированием; инструменты анализа производительности;
			\item инструменты \textbf{сопровождения} ПО: улучшение понимания (напр., визуализация); средства реинженерии;
		\end{itemize}
	}
	
	\frame{
		\frametitle{Инструменты ПИ (продолжение)}
		
		\begin{itemize}
			\item инструменты управления \textbf{конфигурацией}: средства отслеживания дефектов (напр.,~баг-трекеры); 
			системы управления версиями; инструменты сборки, выпуска и~инсталляции;
			\item инструменты \textbf{управления ПИ}: средства для планирования и отслеживания проектов; инструменты управления рисками; инструменты измерения характеристик ПО;
			\item средства \textbf{поддержки процессов} ПИ: инструменты моделирования процессов; инструменты управления процессами; 
			\item инструменты обеспечения \textbf{качества}: средства для аудита и рецензирования; инструменты статического анализа 
			(проверка артефактов на соответствие требованиям).
		\end{itemize}
	}
	
	\section{Качество}
	
	\subsection{Определения}
	
	\frame{
		\frametitle{Качество ПО}

		\begin{Definition}
			\textbf{Качество программного обеспечения} \engterm{software quality} — набор свойств продукта, характеризующих его способность 
			удовлетворить явно заданные или подразумеваемые требования заказчика.
		\end{Definition}

		\vspace{1ex}
		\textbf{Категории} характеристик качества (стандарт ISO 9126:01, \extlink{http://www.iso.org/iso/home/store/catalogue_ics/catalogue_detail_ics.htm?csnumber=35733}{25010:11}):
		\begin{itemize}
			\item внутренние — соответствие промежуточных артефактов внутренним стандартам;
			\item внешние — требования к функциональности продукта;
			\item эксплуатационные \engterm{quality in use} — характеристики качества, интересующие конечного пользователя.
		\end{itemize}
	}
	
	\subsection{Характеристики качества}
	
	\frame{
		\frametitle{Основные характеристики качества}
		
		\begin{itemize}
			\item
			\textbf{Функциональность} \engterm{functional suitability} — степень соответствия продукта явным и~подразумеваемым требованиям 
			при использовании в определенных условиях.

			\vspace{0.5ex}
			\item
			\textbf{Эффективность} \engterm{performance efficiency} — эффективность использования предоставленных ресурсов 
			(напр., оборудования, ОС, других~программ, расходных материалов).

			\vspace{0.5ex}
			\item
			\textbf{Совместимость} \engterm{compatibility} — возможности по обмену информацией с~другими~программами 
			и~совместного оперирования в~одной~среде.

			\vspace{0.5ex}
			\item
			\textbf{Удобство применения} \engterm{usability} — простота обучения, легкость управления системой, 
			доступность пользовательского интерфейса и~т.\,п.
		\end{itemize}
	}

	\frame{
		\frametitle{Основные характеристики качества (продолжение)}
		
		\begin{itemize}
			\item
			\textbf{Надежность} \engterm{reliability} — отказоустойчивость, доступность (\emph{availability}), 
			возможности по~восстановлению после~сбоев (\emph{recoverability}).

			\vspace{0.5ex}
			\item
			\textbf{Безопасность} \engterm{security} — степень защиты данных пользователей.

			\vspace{0.5ex}
			\item
			\textbf{Простота сопровождения} \engterm{maintainability} — эффективность модификации системы, 
			возможности по~повторному~использованию (\emph{reusability}), модульность и~т.\,п.

			\vspace{0.5ex}
			\item
			\textbf{Переносимость} \engterm{portability} — эффективность переноса~ПО в~новую~среду исполнения.
		\end{itemize}
	}

	\subsection{Процессы инженерии качества}
	
	\frame{
		\frametitle{Процессы инженерии качества}

		\begin{itemize}
			\item
			\textbf{Обеспечение} качества \engterm{quality assurance} — деятельность для гарантирования характеристик 
			качества в программном продукте (напр., четкая формулировка требований и проблем, составление планов, …);
			\item
			\textbf{верификация} — обеспечение корректной реализации ПО согласно спецификциям (правильно ли создается система?);
			\item
			\textbf{валидация} — соответствие системы требованиям (удовлетворяет ли система заказчика?);
			\item
			\textbf{инспекции} — выявление аномалий в ПО независимыми экспертами;
			\item
			\textbf{аудит} — независимая оценка продукта на соответствие регламентирующим документам (планам, стандартам и т.\,п.).
		\end{itemize}
	}

	\section{Заключение}
	
	\subsection{Выводы}

	\frame{
		\frametitle{Выводы}

		\begin{enumerate}
			\item
			Ядро SWEBOK содержит пять вспомогательных областей знаний (управление конфигурацией, управление инженерией, 
			процесс инженерии, инженерия качества, а~также методы и~инструменты~ПИ). 
			Они соответствуют организационным и~управленческим аспектам производства~ПО.

			\vspace{0.5ex}
			\item
			Основу организационных областей составляют практические рекомендации.

			\vspace{0.5ex}
			\item
			Организационные области SWEBOK связаны между собой, а~также с~основными областями знаний, для~которых они регламентируют деятельность.
		\end{enumerate}
	}

	\subsection{Материалы}

	\frame{
		\frametitle{Материалы}
		
		\begin{thebibliography}{9}
			\bibitem[1]{1}
			Лавріщева К.\,М. 
			\newblock Програмна інженерія (підручник). 
			\newblock {\footnotesize К., 2008. — 319 с.}

			\bibitem[2]{2}
			IEEE Computer Society
			\newblock Описание стандарта SWEBOK.
			\newblock {\footnotesize\url{http://www.computer.org/portal/web/swebok/html/contents}}
			
			\bibitem[3]{3}
			International Standartization Organization
			\newblock Cтандарты ISO. 
			\newblock {\footnotesize\url{http://www.iso.org/iso/home.html}}
		\end{thebibliography}
	}
	
	\frame{
		\frametitle{}
		
		\begin{center}
			\Huge Спасибо за внимание!
		\end{center}
	}
\end{document}
